\subsection{Logging of Sensor Data}
When a reader detects a object it records a log. 
A reading is of the following format \textless $readerID,objectID,timestamp$ \textgreater.
The \textit{readerID} is the identity of the reader, \textit{objectID} is the identity of the object which is in the range of the reader, and the \textit{timestamp} t is the time where the reader detected the object.
This recording is done at a given sampling rate. 
The same object might be recorded several times by the same reader in a row. 
A reader creates a log for each object that is in the vicinity of the reader thus there can be several logs with the same timestamp.

From this log data on-line and off-line outputs are generated~\cite{Jensen:2009:GMB:1590953.1591000}.
All data from the readers are continuously streamed to a database that is responsible for maintaining the data.
If a object enters or leave the range of a reader an on-line output is generated, this have the form
\begin{align}
\label{log:online}
<readerID,objectID,t,flag >
\end{align}

where \textit{flag} is "`START"' or "`END"' respectively. 
%When the object is no longer detected by the reader an output is generated. 
Note that when a object in no longer detected by the reader  the \textit{flag} is set to "`END"' and the timestamp is the last \textit{t} where the object was within the vicinity.
A second output is generated when it have the form 
\begin{align},
\label{log:offline}
<\textit{$readerID,objectID,t_{first},t_{last}$}>
\end{align}
These two outputs are later used in online and offline tracking which we cover in Section~\ref{sub:online} and Section~\ref{sub:offline}.

 
