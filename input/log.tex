\subsection{log (working title)}

When a reader detects a moving object it records a log. 
A reading is of the following format \textless $readerID,objectID,timestamp$ \textgreater.
The \textit{readerID} is the identity of the reader, \textit{objectID} is the identity of the object which is in the range of the reader, and the \textit{timestamp} is the time where the reader detected the object.
This recording is done at a given sampling rate. 
The same object might be recorded several times by the same reader in a row. 
A reader creates a log for each object that is in the vicinity of the reader thus there can be several logs with the \textit{timestamp}.

From this raw log data two different outputs are created.
All data from the readers are continuously streamed to a database that is responsible for maintaining the data.
If a object enters the range of a reader an output is generated, this have the form \textless $readerID,objectID,timestamp,flag$ \textgreater, where \textit{flag} is "`START"'. 
When the object is no longer detected by the reader an output with the previous \textit{timestamp} and the \textit{flag} set to "`END"' it outputted.
The second output is generated based on the first output. 
It have the format \textless\textit{$readerID,objectID,timestamp_{first}$,$timestamp_{last}$}\textgreater .


 
