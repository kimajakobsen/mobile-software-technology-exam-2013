%!TEX root = ../bare_adv.tex
\subsection{Logging of Sensor Data}
When a reader detects an object it records it in a log. 
A reading is of the following format $\langle readerID,objectID,t\rangle$.
The \textit{readerID} is the identity of the reader, \textit{objectID} is the identity of the object which is in the range of the reader, and $t$
is the timestamp where the reader detected the object.
This recording is done at a given sampling rate. 
The same object might be recorded several times by the same reader in a row. 
A reader creates a log entry for each object that is in the vicinity of the reader thus there can be several log entries with the same timestamp.

From this log data on-line and off-line outputs are generated~\cite{Jensen:2009:GMB:1590953.1591000}.
All data from the readers are continuously streamed to a database that is responsible for maintaining the data.
If an object enters or leaves the range of a reader an on-line output is generated, this have the form
\begin{align}
\label{log:online}
\langle readerID,objectID,t,flag \rangle
\end{align}

where \textit{flag} is ``START'' or ``END'' respectively. 
%When the object is no longer detected by the reader an output is generated. 
Note that when an object in no longer in vicinity of the reader  the $flag$ is set to ``END'' and the $t$ is the last timestamp where the object was within vicinity.
A second output is then generated with the form 
\begin{align}
\label{log:offline}
\langle readerID,objectID,t_{first},t_{last} \rangle
\end{align}
Where $t_{first}$ and $t_{last}$ is the timestamp of the device entry and exit of the readers vicinity.
The last two outputs ((\ref{log:online}) and (\ref{log:offline})) are later used in online and offline tracking which we cover in Section~\ref{sub:online} and Section~\ref{sub:offline}.

 
