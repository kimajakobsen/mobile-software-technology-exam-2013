
\documentclass[12pt,journal,compsoc]{IEEEtran}

\usepackage{amsmath}
\usepackage{graphics}
\usepackage{float}

% *** CITATION PACKAGES ***
%
\ifCLASSOPTIONcompsoc
  % IEEE Computer Society needs nocompress option
  % requires cite.sty v4.0 or later (November 2003)
  % \usepackage[nocompress]{cite}
\else
  % normal IEEE
  % \usepackage{cite}
\fi


% *** GRAPHICS RELATED PACKAGES ***
%
\ifCLASSINFOpdf
  % \usepackage[pdftex]{graphicx}
  % declare the path(s) where your graphic files are
  % \graphicspath{{../pdf/}{../jpeg/}}
  % and their extensions so you won't have to specify these with
  % every instance of \includegraphics
  % \DeclareGraphicsExtensions{.pdf,.jpeg,.png}
\else
  % or other class option (dvipsone, dvipdf, if not using dvips). graphicx
  % will default to the driver specified in the system graphics.cfg if no
  % driver is specified.
  % \usepackage[dvips]{graphicx}
  % declare the path(s) where your graphic files are
  % \graphicspath{{../eps/}}
  % and their extensions so you won't have to specify these with
  % every instance of \includegraphics
  % \DeclareGraphicsExtensions{.eps}
\fi

\providecommand{\PSforPDF}[1]{#1}


% NOTE: PDF hyperlink and bookmark features are not required in IEEE
%       papers and their use requires extra complexity and work.
% *** IF USING HYPERREF BE SURE AND CHANGE THE EXAMPLE PDF ***
% *** TITLE/SUBJECT/AUTHOR/KEYWORDS INFO BELOW!!           ***
\newcommand\MYhyperrefoptions{bookmarks=true,bookmarksnumbered=true,
pdfpagemode={UseOutlines},plainpages=false,pdfpagelabels=true,
colorlinks=true,linkcolor={black},citecolor={black},pagecolor={black},
urlcolor={black},
pdftitle={Bare Demo of IEEEtran.cls for Computer Society Journals},%<!CHANGE!
pdfsubject={Typesetting},%<!CHANGE!
pdfauthor={Michael D. Shell},%<!CHANGE!
pdfkeywords={Computer Society, IEEEtran, journal, LaTeX, paper,
             template}}%<^!CHANGE!



% correct bad hyphenation here
\hyphenation{op-tical net-works semi-conduc-tor}


\begin{document}
%
% paper title
% can use linebreaks \\ within to get better formatting as desired
\title{Survey of Indoor Positioning Techniques}
%
%


\author{Rasmus~Prentow,~\IEEEmembership{Student,}
        Anders~Eiler,~\IEEEmembership{Student,}
        and~Kim~Jakobsen,~\IEEEmembership{Student}% <-this % stops a space
\IEEEcompsocitemizethanks{\IEEEcompsocthanksitem M. Shell is with the Department
of Electrical and Computer Engineering, Georgia Institute of Technology, Atlanta,
GA, 30332.\protect\\
% note need leading \protect in front of \\ to get a newline within \thanks as
% \\ is fragile and will error, could use \hfil\break instead.
E-mail: see http://www.michaelshell.org/contact.html
%\IEEEcompsocthanksitem J. Doe and J. Doe are with Anonymous University.
}% <-this % stops a space
\thanks{Manuscript received April 19, 2005; revised January 11, 2007.}}





% The paper headers
\markboth{Journal of \LaTeX\ Class Files,~Vol.~6, No.~1, January~2007}%
{Shell \MakeLowercase{\textit{et al.}}: Bare Advanced Demo of IEEEtran.cls for Journals}
% The only time the second header will appear is for the odd numbered pages
% after the title page when using the twoside option.
% 
% *** Note that you probably will NOT want to include the author's ***
% *** name in the headers of peer review papers.                   ***
% You can use \ifCLASSOPTIONpeerreview for conditional compilation here if
% you desire.






% for Computer Society papers, we must declare the abstract and index terms
% PRIOR to the title within the \IEEEcompsoctitleabstractindextext IEEEtran
% command as these need to go into the title area created by \maketitle.
\IEEEcompsoctitleabstractindextext{%
\begin{abstract}
%\boldmath
The abstract goes here.
\end{abstract}
% IEEEtran.cls defaults to using nonbold math in the Abstract.
% This preserves the distinction between vectors and scalars. However,
% if the journal you are submitting to favors bold math in the abstract,
% then you can use LaTeX's standard command \boldmath at the very start
% of the abstract to achieve this. Many IEEE journals frown on math
% in the abstract anyway. In particular, the Computer Society does
% not want either math or citations to appear in the abstract.

% Note that keywords are not normally used for peerreview papers.
%\begin{IEEEkeywords}
%Computer Society, IEEEtran, journal, \LaTeX, paper, template.
%\end{IEEEkeywords}}
}

% make the title area
\maketitle


% To allow for easy dual compilation without having to reenter the
% abstract/keywords data, the \IEEEcompsoctitleabstractindextext text will
% not be used in maketitle, but will appear (i.e., to be "transported")
% here as \IEEEdisplaynotcompsoctitleabstractindextext when compsoc mode
% is not selected <OR> if conference mode is selected - because compsoc
% conference papers position the abstract like regular (non-compsoc)
% papers do!
\IEEEdisplaynotcompsoctitleabstractindextext
% \IEEEdisplaynotcompsoctitleabstractindextext has no effect when using
% compsoc under a non-conference mode.


% For peer review papers, you can put extra information on the cover
% page as needed:
% \ifCLASSOPTIONpeerreview
% \begin{center} \bfseries EDICS Category: 3-BBND \end{center}
% \fi
%
% For peerreview papers, this IEEEtran command inserts a page break and
% creates the second title. It will be ignored for other modes.
\IEEEpeerreviewmaketitle


%!TEX root = ../bare_adv.tex
\section{Introduction}
% paper discusses different technologies for wireless indoor positioning and tracking. 
\IEEEPARstart{D}{uring} the past decade, technologies like GPS have developed so much, that it is possible to track a person or an object within a few meters of is position. 
Unfortunately is the frequency in which GPS signals are transmitted easily blocked by buildings and other structures. 
Therefore are a new field of study needed to fill in this cap such that positions can be established in places where GPS is unavailable.
In this paper we look at the field of indoor positioning. 
The reason indoor positioning is interesting is that most of the time we are indoor, at work or at home. 
Activities such as shopping are today often occurring in malls, which have interesting uses for indoor positing since tracking of customers whereabouts can help placing advertisements and prices of store floor. This can also help to value a store, since stores placed in areas with high population are more valuable than stores placed in a low-density area.
When looking at indoor positioning, there are a number of elements that makes it more difficult to track a person or an object compared to outdoor.
%The GPS signals that are used for outdoor positioning are too weak to be used indoor. 
%Also the structure of buildings (walls, several levels, doors etc.) makes it further complicated.
As GPS cannot be used other methods for indoor tracking and positioning has to be investigated. 
These methods include Graph Based Indoor Tracing and WI-FI based services, which we explore in this paper along hybrids thereof.  


% demo file is intended to serve as a ``starter file''
%for IEEE Computer Society journal papers produced under \LaTeX\ using
%IEEEtran.cls version 1.7 and later.
% You must have at least 2 lines in the paragraph with the drop letter
% (should never be an issue)
%I wish you the best of success.




\section{Graph Based Indoor Tracking}
%!TEX root = ../bare_adv.tex
%\subsection{Introduction}
Graph-based Modelling~\cite{Jensen:2009:GMB:1590953.1591000} can be used to track the location of objects. 
It uses plans of indoor areas as graphs e.g. as a floor plan of the indoor area. 
This area may be a complex area with several rooms, levels, doors, hallways etc. 
The connectivity and accessibility of the area are represented in as a graph where each room is a vertex and each connection as an edge. 

By representing areas this way, it is possible to apply indoor track people in the areas from the floor plan, this is done using various low-proximity technologies as Bluetooth, Wi-Fi, RFID etc.

When using a low-proximity technology, the data that is recorded by the sensors, regardless of the technique that is used for this, is saved in a database for later querying and analysis. 
This allows for both Online and Offline tracking.
Online being live querying of the currently available data to see the location of the object, and Offline being a later analysis of the recorded data, if you e.g. want to track the movement pattern of one or more persons. 
To perform online or offline tracking need to three elements: A Deployment Graph, processed sensor data, and the max movement speed of the targeted object.
We now explore these three elements.
This section is largely based on~\cite{Jensen:2009:GMB:1590953.1591000}.

%A Graph-based Model can be applied, as it does not differ between the actual floor plan and the abstract model, hence why using an abstract model provides a equally correct positioning data. 

\subsection{Connectivity, Accessibility, \& Deployment Graphs}
There are three graphs that describes the the topology of a floor plan. 
The first is the \textit{Connectivity Graph} which describes how rooms, hallways, stairs etc. are connected to each other. 
The second is the \textit{Accessibility Graph} which takes the actual accessibility of a cell into account. 
The third is the \textit{Deployment Graph} which contains the rooms and doorways that are equipped with scanners.


\subsubsection{ \quad Connectivity Graph}
Both the Connectivity graph and the Accessibility Graph are constructed based on the actual floor plan. 

\begin{figure}[]%
\centering
\includegraphics[width=0.8\columnwidth]{images/floorplan.png}%
\caption{Floor plan of a building. Inspired by~\cite{Jensen:2009:GMB:1590953.1591000}}%
\label{fig:floortplan}%
\end{figure}%

Figure \ref{fig:floortplan} shows the floor plan, that is used in our examples.
It contains 7 cells, labelled $R1 - R7$, one staircase labelled $S1$, a hallway and 12 doors labelled $d1-d12$. \\
The area which a cell covers is denoted $A_{cell}$.

In the Connectivity Graph, each separate partitioning of the floor plan (rooms, staircases, hallways etc.) are represented as vertexes, while the connection of these (doors, windows, hatches etc.) are represented as the edges.
E.g. if two rooms are connected by a door, each cell is represented in the base-graph as a vertex, while the door is represented as an undirected edge. 
The Connectivity Graph representing the floor plan from Figure \ref{fig:floortplan} is shown in Figure \ref{fig:connectivitygraph}. 
\begin{figure}[]%
\centering
\includegraphics[width=0.8\columnwidth]{images/connectivitygraph.png}%
\caption{Connectivity Graph of the floor plan in Figure \ref{fig:floortplan}.}%
\label{fig:connectivitygraph}%
\end{figure}%
Figure \ref{fig:connectivitygraph} shows how the rooms and hallways in Figure \ref{fig:floortplan} are connected by doors. 
The Connectivity Graph itself is a labelled, undirected multi-graph that is defined by the following triple: \\
\begin{equation}
G_{connection} = (V, E_d, \Sigma_{door})
\end{equation}
In this triple, $V$ is the set of vertexes in the Graph. 
$E_d$ is the set of edges in the Graph, where any edge in $E_d$ is a set consisting of $(\{v_i, v_j\}, k)$ such that $v_i, v_j \in V$ and $k \in \Sigma_{door}$.
Finally, $\Sigma_{door}$ is a set of edge labels that represent the connections. 

%However, it does not make much sense to look at the Connectivity Graph alone when talking about indoor positioning and tracking. 
%To be able to track movement, information about the accessibility of each vertex must also be available. 

\subsubsection{ \quad Accessibility Graph}
There may be situations where a connection does not mean that you have access through that connection.
It can be air-port security or subway ententes that are either entries or exits, which allows one-way movement only.
To take this into account, the Accessibility Graph is used. 
The Accessibility Graph for the floor plan from Figure \ref{fig:floortplan} is shown in Figure \ref{fig:accesibbilitygraph}.
\begin{figure}[]%
\centering
\includegraphics[width=0.8\columnwidth]{images/accessibilitygraph.png}%
\caption{Accesibility graph of the floor plan in Figure \ref{fig:floortplan}.} % and the connectivity graph in Figure \ref{fig:floortplan}.}%
\label{fig:accesibbilitygraph}%
\end{figure}%
Figure \ref{fig:accesibbilitygraph} shows the Accessibility Graph.
It can be seen that door $d10$ only allows entrance from outside and into the hall, while door $d11$ only allows exit from the hall.
The Accessibility Graph is a labelled, directed graph and constructed to represent available movement patterns in the area. \\
An Accessibility Graph $G_{access}$ is given by the triple: 
\begin{equation}
G_{access} = (V, E, \Sigma_{door}, l_e)
\end{equation} 
where $V$ is the set of vertexes, $E$ is the set of directed edges, having $E = \{\langle v_i, v_j \rangle | v_i, v_j \in V,  v_i \not= v_j\}$ and finally $l_e$ is a function that maps edges to subsets of the doors, having $l_e : E \rightarrow 2^{\Sigma_{door}}$. \\

%These graphs provide an abstract representation of the floor plan, which -- combined with the previously mentioned logging -- will allow for indoor movement tracking and indoor positioning. 

\subsubsection{ \quad Deployment Graph}
While the Accessibility Graph is based on the Connectivity Graph, the Deployment Graph is based on the Accessibility Graph. 
The Deployment Graph is an Accessibility graph where edges is doors with sensors e.g. if two rooms do not have a sensor at the passage door then the two rooms are view as the same room.
%When it is known in what direction people can travel, it is possible to make deployment plan for the sensors to read movement patterns. 
%This scenario is based on RFID technology that uses proximity analysis for determining when a tag is approaching a reader. 
%All RFID readers must have disjoint activations ranges, meaning that a tag cannot be read by more than one reader at the time. \\
The actual deployment varies from case to case. 
There are two kind of readers. 
Partitioning readers and Presence readers. \\

The partitioning readers partition the indoor space into different cells. 
An object cannot move from one cell to another without being read by a sensor. 
When deploying partitioning readers, one can choose between placing one or two readers in every door opening. 
By placing one reader only, it is not possible to track which way the object is moving - only that is has been recorded passing the reader in question. 
%In order to find out in which way the object was moving, the history of other reads of the same object has to be taken into account. 
By placing two readers in every door opening, it is possible to detect in which direction the object is moving (e.g. if it is moving in or out of the cell). \\

Presence readers are readers that do not contribute to the partitioning of the indoor space. 
They simply observe which objects are present within its range. 
%These recordings will also have an entry- and an exit time of when the any given object enters or exists the readers range. \\

A deployment graph should be complete, meaning that it should capture all present cells and connections between them, and it should have as few vertices and edges as possible. \\
An RFID deployment graph is represented as a directed, labelled graph:
\begin{equation}
G_{RFID} = (C, E_r, \Sigma_{reader}, l_e)
\end{equation} 
where $C$ is the set of vertixes, $E_r$ is the set of edges (an edge is an ordered pair $\langle c_i, c_j \rangle$ of distinct vertexes from $C$) and $l_e : E_r \rightarrow 2^{\Sigma_{reader}} \cup 2^{\Sigma_{reader} X \Sigma_{reader}}$ maps an edge to a partitioning reader or a partitioning reader pair. \\

The Deployment Graph is a map of how readers are located in an indoor area. 

%\IEEEPARstart{T}{his} demo file is intended to serve as a ``starter file''
%for IEEE Computer Society journal papers produced under \LaTeX\ using
%IEEEtran.cls version 1.7 and later.
% You must have at least 2 lines in the paragraph with the drop letter
% (should never be an issue)
%I wish you the best of success.
%\hfill mds
%\hfill January 11, 2007

\section{Introduction}

\IEEEPARstart{T}{his} demo file is intended to serve as a ``starter file''
for IEEE Computer Society journal papers produced under \LaTeX\ using
IEEEtran.cls version 1.7 and later.
% You must have at least 2 lines in the paragraph with the drop letter
% (should never be an issue)
I wish you the best of success.

\hfill mds
 
\hfill January 11, 2007

\subsection{Maximum Speed}
When an object is in a vacant time interval its position can be narrowed further down then the cell it is presumer to be located in. 
This is done based on the max speed of the object which is $V_{max}*\Delta t_1$ . 
We view the range of the sensor $s$ as a circle with the sensors maximum range as radious, we call this $r_1$. 
Knowing the maximum speed of the object a circle can be drawn from the point where the object lef the vacinity of the sensor.
This circle is the area where the object is located. 
It is, however, not know the point of exit from the sensor but only that the object left. 
The area of the object can then be calculated by creating a circle with the maximum sensor range plus the maximum range that the object can have traveled. 





\begin{figure}%
\includegraphics{images/speed.png}%
\caption{}%
\label{}%
\end{figure}

\subsection{Offline Tracking}
\label{sub:offline}
With the information from the above subsections we are now able to produce a trajectory of a given object.
An offline trajectory contain data of where the object have already moved and is displayed in the following form:
\begin{align}
<readerID,objectID,tagID,t_{start},t_{end}>
\end{align}
To transform the data into this form we perform a three step refinement.
First the result of Output \ref{log:offline} is combined with the deployment graph.
We extract all entries with the matching objectID, each entry is mapped to the deployment graph.
The result of this is a table where edges and cells from the deployment graphs is shown, see Figure \ref{fig:ref1}.  
Next we revers the data such that we have the time intervals where object are not in sensor range.
In Figure \ref{fig:ref2} the three sensor reading are converted into two intervals where we know which cell the object is located in.
Last we further limit the area of which the object is located in. 
We use the knowledge obtained in Section \ref{sec:speed} to limit the area based on max speed.
 

\begin{tabular}{ l | c | r | }
\hline
   objectID & Interval & Location\\ 
\hline
Object1 & [$t_1,t_2$] & c9 \\ 
\hline
Object1 & [$t_6,t_10$] &  $e<9,3>$\\ 
\hline
Object1 & [$t_21,t_30$] &  $e<3,9>$\\ 
\hline
\caption{Example of refinement steps one. First row is a reading done by non-directional sensor, the second and third row is a reading from a directional sensor.}
\label{fig:ref1}
\end{tabular}

\begin{tabular}{ l | c | r | }
\hline
   objectID & Interval & Location\\ 
\hline
Object1 & [$t_3,t_5$] & c9 \\ 
\hline
Object1 & [$t_11,t_20$] &  c3\\ 
\hline
\caption{Example of refinement steps two.}
\end{tabular}
   


\subsection{Online Tracking}
\label{sub:online}






% The very first letter is a 2 line initial drop letter followed
% by the rest of the first word in caps (small caps for compsoc).
% 
% form to use if the first word consists of a single letter:
% \IEEEPARstart{A}{demo} file is ....
% 
% form to use if you need the single drop letter followed by
% normal text (unknown if ever used by IEEE):
% \IEEEPARstart{A}{}demo file is ....
% 
% Some journals put the first two words in caps:
% \IEEEPARstart{T}{his demo} file is ....
% 
% Here we have the typical use of a "T" for an initial drop letter
% and "HIS" in caps to complete the first word.



\section{Wi-Fi based location service}
\label{wififingerprinting}
Wi-Fi is a widespread technology and is deployed at most offices, homes, and public buildings.
It is used in indoor positioning systems \cite{mlws,5388848,radarlf,Bell2010,6068444} and provides a low installtion cost as no extra infrastructure requirements are needed. 
If Wi-Fi signals propagated trough space, such that distance to the emitter and the receiver is correlated, utilizing Wi-Fi to determine a devices position would be trivial~\cite{ariadne2006}. 
It could be solved by having a minimum of three Wi-Fi radios with known positions. 
Finding the devices position would be a matter of solving a quadratic equation with two unknowns.  
Unfortunately Wi-Fi signals are greatly affected by building interior, people moving trough the space, humidity, and other factors making it nontrivial to use for location based services~\cite{6068444}.
It is expected that using solely Wi-Fi gives an estimation accuracy between 2 to 10 meters~\cite{6068444,mlws}

A class of techniques which can utilizes Wi-Fi is the \textit{location fingerprinting}(LF) techniques~\cite{taxonomy2007}. 
An LF technique is divided into two phases. 
An initial phases, called the offline phase, in which a \textit{radio map} or is created and a second phase where the radio map is used to estimate the device's location.
The two phases are elaborated in the following two sections. 

\subsection{Generating a Radio Map}
A radio map can be created using either an empirical or a model-based approach~\cite{taxonomy2007}. 
A model-based approach uses parameterized model for LF based on the building layout and a theoretical propagation model~\cite{radarlf}.   
Empirical methods can be subdivided into probabilistic and deterministic models.  
Both use measured data, known as fingerprints, often collected by researchers wandering around the area to be mapped with a laptop performing measurements~\cite{roblocal2004,5388848, radarlf}.  
Deterministic models treat the measured fingerprints as single values while probabilistic methods represent data as probability models. 
Both empirical methods can be divided into interpolating and aggregating methods. 
Aggregating methods combine measured values into mean values in the radio map~\cite{radarlf, roblocal2004}, while interpolating methods creates new data points by interpolating existing data points~\cite{locadio2004}. 
Deterministic models utilize direct-methods where fingerprint data are used without any further processing or with outliers removed, as in~\cite{1200692}. 

\subsection{Estimating Positions}
Methods for estimating the position of the tracked device are, as empirical methods for radio map generation, divided into the two subcategories: deterministic and probabilistic~\cite{taxonomy2007}.
There are several deterministic methods, but most is variations to NN or KNN algorithms as
the nearest neighbor in signal space method (NNSS)
described in~\cite{6068444} and \cite{radarlf}.
Assume that there are $n$ Wi-Fi points in the indoor space of interest and $m$ of those are pre-selected as reference points. 
A signal measure or fingerprint is then an $n$-dimensional vector $s=(s_{1},\dots,s_{n})$ and $s_{i} (1 \le  i \le n)$ denotes the signal strength from the $i$-th Wi-Fi radio. 
The distance is calcualted as euclidian distance the current signal measure (query) $q$ and each measured fingerprint $s^{j} (1 \le j \le m)$, 
that is $dist(q,s^{j}) = \sqrt{\sum_{i=1}^{n}\left(q_{i}-s^{j}_{i}\right)}$ 
Given a radio map and a user signal strength measure, the location can be found by searching the radio map. 
If there is an exact match the user is at this location otherwise NN query can be used.
As described in~\cite{ariadne2006} using KNN is preferred to single NN as the query point is often close too two or more radio map entries.
Probabilistic methods include hidden markov chains and center of mass~\cite{Youssef2005}.









\section{Hybrid Techniques}
So far we have discovered Graph-based solutions, using close proximity devices, such as RFID, NFC, and Bluetooth and Fingerprinting techniques using Wi-Fi. 
Combining the Wi-Fi based fingerprinting with low proximity technology such as Bluetooth has been done with good results~\cite{6068444}.
As mentioned in Section~\ref{wififingerprinting} Wi-Fi solutions have several issues regarding accuracy, but it has low infrastructure cost -- as it is often already deployed -- and high range. 
Bluetooth technology is quite the opposite as it is rarely deployed in buildings, it has a short range. 
The short range ensure that positions can be estimated fairly accurately as whenever a device can read the signal from a Bluetooth transmitter it is physically located close to transmitter. 
Bluetooth transmitter have varying propagation distances, but it is preferred for the technique explained in the following that the range is short. 
The technique which we have investigated are described in~\cite{6068444}.

%\subsection{Basic Concept}  
The idea is to partition the indoor space into regions using Bluetooth transmitters.
An example of such an partition is shown in Figure \ref{fig:partionedcluster}. 
Here position 5, 7, and, 19 contains a Bluetooth transmitter that the user cannot pass without the device he uses for position determination registers the sender.  
%The cluster in Figure \ref{fig:partionedcluster} contains 4 regions. 

\begin{figure}%
\includegraphics[width=\columnwidth]{images/partionedcluster}%
\caption{An indoor space partitioned into four region. Adapted from~\cite{6068444}.}%
\label{fig:partionedcluster}%j
\end{figure}%   
The partitioning is done at narrow places, such as hallways and doorways. 
This is necessary to ensure that a user cannot pass through the Bluetooth transmitter without noticing it. 
Once a user passes a Bluetooth transmitter in an indoor space we can limit his possible position to two regions. 
E.g. if a user is registered at the transmitter in location 7 in Figure \ref{fig:partionedcluster} he can be in either region 2 or 3. 
Until the user passes another Bluetooth transmitter he can be assumed to be in these two regions.
If he for instances passes the transmitter in location 5, he can be in either region 1 or 2. 
It is, however, not possible to determine the moving direction of the sender because he may have entered the proximity of the transmitter and then returned from where he came. 

%\subsection{Device and Architecture}
%The device is typically a smartphone (during the testing a small laptop is used) or similar with an application installed. 
%The application registers the Bluetooth and Wi-Fi signals strengths with $1/3$Hz and $1$Hz respectively.  
%The signal strengths are send to a server which calculates the position of the device. 
%The entire system architecture can be seen in Figure~\ref{fig:hybridsystemarchitecture}.
\begin{figure}%
    \includegraphics[width=\columnwidth]{images/hybridsystemarchitecture.png}%
\caption{The system architecture of the hybrid system. Adapted from~\cite{6068444}.}%
\label{fig:hybridsystemarchitecture}%
\end{figure}
%\subsection{Results}
From the test performed by the research team the hybrid technique improves the overall accuracy compared to purely Wi-Fi based techniques. 
Figure \ref{fig:hybridsystemarchitecture} shows the architecture of the tested system.
%Due to the partitioning of the indoor space, the search area for a query is very limited.
%Thus making each query perform relatively fast. %In~\cite{
  



\section{Discussion}

%%%%NOTES%%%
%what happens if a reader fails
%pros and cons for each method
%point our that they can be combined







%\appendices

%%%%%%%%%%%%%%%%%
%Make real bibtex setup
%%%%%%%%%%%%%%%%%


%Appendix two text goes here.


% use section* for acknowledgement
%\ifCLASSOPTIONcompsoc
  % The Computer Society usually uses the plural form
%jj  \section*{Acknowledgments}
%\else
  % regular IEEE prefers the singular form
%  \section*{Acknowledgment}
%\fi


The authors would like to thank...


% Can use something like this to put references on a page
% by themselves when using endfloat and the captionsoff option.
%\ifCLASSOPTIONcaptionsoff
%  \newpage
%\fi
%




\bibliographystyle{amsplain}
\bibliography{bib}

% biography section
% 
% If you have an EPS/PDF photo (graphicx package needed) extra braces are
% needed around the contents of the optional argument to biography to prevent
% the LaTeX parser from getting confused when it sees the complicated
% \includegraphics command within an optional argument. (You could create
% your own custom macro containing the \includegraphics command to make things
% simpler here.)
%\begin{biography}[{\includegraphics[width=1in,height=1.25in,clip,keepaspectratio]{mshell}}]{Michael Shell}
% or if you just want to reserve a space for a photo:

%\begin{IEEEbiography}{Michael Shell}
%Biography text here.
%\end{IEEEbiography}
%
%% if you will not have a photo at all:
%\begin{IEEEbiographynophoto}{John Doe}
%Biography text here.
%\end{IEEEbiographynophoto}
%
%% insert where needed to balance the two columns on the last page with
%% biographies
%%\newpage
%
%\begin{IEEEbiographynophoto}{Jane Doe}
%Biography text here.
%\end{IEEEbiographynophoto}
%
% You can push biographies down or up by placing
% a \vfill before or after them. The appropriate
% use of \vfill depends on what kind of text is
% on the last page and whether or not the columns
% are being equalized.

%\vfill

% Can be used to pull up biographies so that the bottom of the last one
% is flush with the other column.
%\enlargethispage{-5in}



% that's all folks
\end{document}


