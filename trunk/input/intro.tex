%!TEX root = ../bare_adv.tex
\section{Introduction}
% paper discusses different technologies for wireless indoor positioning and tracking. 
\IEEEPARstart{D}{uring} the past decade, technologies like GPS have developed so much, that it is possible to track a person or an object within a few meters of is position. 
Unfortunately is the frequency in which GPS signals are transmitted easily blocked by buildings and other structures. 
Therefore are a new field of study needed to fill in this cap such that positioning can be established in places where GPS is unavailable.
In this paper we look at the field of indoor positioning. 
The reason indoor positioning is interesting is that most of the time we are indoor, at work or at home. 
Activities such as shopping are today often occurring in malls, which have use for indoor positioning since tracking of customers whereabouts can help placing advertisements. 
This can also help to value a store, since stores placed in areas with high population are more valuable than stores placed in a low-density area.
When looking at indoor positioning, there are a number of elements that makes it more difficult to track a person or an object compared to outdoor.
%The GPS signals that are used for outdoor positioning are too weak to be used indoor. 
%Also the structure of buildings (walls, several levels, doors etc.) makes it further complicated.
As GPS cannot be used other methods for indoor tracking and positioning have to be investigated. 
Two of the most promising fields are Graph Model Based Indoor Tracking and fingerprinting, which we explore in this paper along hybrids thereof. 
%We explain Graph Based Indoor Tracking and explain the concept o  


% demo file is intended to serve as a ``starter file''
%for IEEE Computer Society journal papers produced under \LaTeX\ using
%IEEEtran.cls version 1.7 and later.
% You must have at least 2 lines in the paragraph with the drop letter
% (should never be an issue)
%I wish you the best of success.


