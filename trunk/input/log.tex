\subsection{log (working title)}

When a reader detects a moving object it records a log. 
A reading is of the following format \textless $readerID,objectID,timestamp$ \textgreater.
The \textit{readerID} is the identity of the reader, \textit{objectID} is the identity of the object which is in the range of the reader, and the \textit{timestamp} t is the time where the reader detected the object.
This recording is done at a given sampling rate. 
The same object might be recorded several times by the same reader in a row. 
A reader creates a log for each object that is in the vicinity of the reader thus there can be several logs with the same \textit{timestamp}.

From this log data two different outputs are created.
All data from the readers are continuously streamed to a database that is responsible for maintaining the data.
If a object enters the range of a reader an output is generated, this have the form
\begin{equation}
\textbf{On-line output:}
 \textless readerID,objectID,t,flag \textgreater
\end{equation} 
where \textit{flag} is "`START"'. 
When the object is no longer detected by the reader an output is generated. 
It contains the last \textit{t} where the object was within the vecinity and the \textit{flag} set to "`END"'.
A second output is generated based on the first output. 
It have the format 
\begin{align}
%\textbf{Off-line output:}\\
\textless\textit{$readerID,objectID,t_{first},t_{last}$}\textgreater
\end{align}


 
