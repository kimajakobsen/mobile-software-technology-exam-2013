\subsection{Offline Tracking}
\label{sub:offline}
With the information from the above subsections we are now able to produce a trajectory of a given object.
An offline trajectory contain data of where the object have already moved and is displayed in the following form:
\begin{align}
<readerID,objectID,tagID,t_{start},t_{end}>
\end{align}
To transform the data into this form we perform a three step refinement.
First the result of Output \ref{log:offline} is combined with the deployment graph.
We extract all entries with the matching objectID, each entry is mapped to the deployment graph.
The result of this is a table where edges and cells from the deployment graphs is shown, see Figure \ref{fig:ref1}.  
Next we revers the data such that we have the time intervals where object are not in sensor range.
In Figure \ref{fig:ref2} the three sensor reading are converted into two intervals where we know which cell the object is located in.
Last we further limit the area of which the object is located in. 
We use the knowledge obtained in Section \ref{sec:speed} to limit the area based on max speed.
 

\begin{tabular}{ l | c | r | }
\hline
   objectID & Interval & Location\\ 
\hline
Object1 & [$t_1,t_2$] & c9 \\ 
\hline
Object1 & [$t_6,t_10$] &  $e<9,3>$\\ 
\hline
Object1 & [$t_21,t_30$] &  $e<3,9>$\\ 
\hline
\caption{Example of refinement steps one. First row is a reading done by non-directional sensor, the second and third row is a reading from a directional sensor.}
\label{fig:ref1}
\end{tabular}

\begin{tabular}{ l | c | r | }
\hline
   objectID & Interval & Location\\ 
\hline
Object1 & [$t_3,t_5$] & c9 \\ 
\hline
Object1 & [$t_11,t_20$] &  c3\\ 
\hline
\caption{Example of refinement steps two.}
\end{tabular}
   


\subsection{Online Tracking}
\label{sub:online}

