%!TEX root = ../bare_adv.tex
\subsection{Maximum Speed}
\label{sec:speed}
When an object is outside the range of sensors it is said to be vacant.
When an object is in a vacant time interval its position can be narrowed down to the cell it is located in, see Figure \ref{ig:vacant}. 
Further it can be narrowed down based on the maximum speed of the object denoted $V_{max}\cdot\Delta t_1$ . 
The speed of an object type may vary, e.g. a human have a maximum recorded speed of 37.58 km/h~\cite{bolt}.
We view the range of the sensor $P_1$ as a circle with the sensors maximum range as radius, we denote this $R_1$, see Figure \ref{fig:speed1}.
Knowing the maximum speed of the object a circle can be drawn from the point where the object left the vicinity of the sensor, see $R_4$.
By drawing a circle with the radius $R_1 + R_4$, see $R_5$, is the area where the object is located. 
The next reading of a sensor of the object can be used to create a oval between the two sensor readings, see Figure \ref{fig:speed2}.  

\begin{figure}%
\centering
\includegraphics[width=\columnwidth]{images/vacant.png}%
\caption{The x-axis show the timestamps where the object enters the vicinity of a sensor. The colored areas are the time interval where the object in within the range of the sensor. The figure is take from~\cite{Jensen:2009:GMB:1590953.1591000}.}%
\label{fig:vacant}%
\end{figure}

\begin{figure}%
\centering
\includegraphics[width=0.5\columnwidth]{images/speed.png}%
\caption{The area that the object can move with in $V_{max}*\Delta t_1$. The figure is adapted from~\cite{Jensen:2009:GMB:1590953.1591000}.}%
\label{fig:speed1}%
\end{figure}

\begin{figure}%
\centering
\includegraphics[width=0.8\columnwidth]{images/speed2.png}%
\caption{This figure illustrates the area where the object can be located in the vacant time between the two sensor reading. The figure is take from~\cite{Jensen:2009:GMB:1590953.1591000}.}%
\label{fig:speed2}%
\end{figure}





