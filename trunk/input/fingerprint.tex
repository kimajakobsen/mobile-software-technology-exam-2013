
\section{Wi-Fi based location service}
Wi-Fi is a widespread technology and is deployed at most offices, homes,and pulic buildings.
It is often used in indoor positioning systems \cite{mlws,5388848,radarlf,Bell2010,6068444} as no extra infrastructure requirements are needed. 
If Wi-Fi signals propagated trough space, such that distance to the emitter and the receiver is correlated, utilizing Wi-Fi to determine a devices position would be trivial~\cite{ariadne2006}. 
It could be solved by having a minimum of three emitters with known positions. 
Finding the devices position would be a matter of solving a quadratic equation with two unknowns.   
Unfortunately Wi-Fi signals are greatly affected by building interior, people moving trough the space, and other factors making it nontrivial to use for location based services. 

A class of techniques which can utilize Wi-Fi is \textit{location fingerprinting}~\cite{taxonomy2006}. 
The techniques are divided into two phases. 
An initial phases, called the offline phase, in which a \textit{radio map} is created and a second phase where the radio map is used to estimate the device's location. 

\subsection{Generating a Radio Map}
A radio map can be created using either an empirical or a model-based approach~\cite{taxonomy2008}. 
A model-based approach use parameterized model for LF based on the building layout and a theoretical propagation model~\cite{radarlf}.   
Empirical models can be subdivided into probabilistic and deterministic models.  
Both use measured data, known as fingerprints, often collected by researchers wandering around the area to be mapped with a laptop performing measurements~\cite{roblocal2004,5388848, radarlf}.  
Deterministic models treat the measured fingerprints as single values while probabilistic methods represent data as probability models. 
Both empirical methods can be divided into interpolating and aggregating methods. 
Aggregating methods combine measured values into mean values in the radio map~\cite{radarlf, roblocal2004}, while interpolating creates new data points by interpolating existing data points~\cite{locadio2004}. 

\subsection{Estimating Positions}
